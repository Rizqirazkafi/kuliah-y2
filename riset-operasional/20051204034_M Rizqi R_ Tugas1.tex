\documentclass{article}

\usepackage{amsmath}
\usepackage{graphicx}
\usepackage{hyperref}
\usepackage[utf8]{inputenc}
\usepackage{lipsum}
\usepackage{multicol}
\usepackage{titling}
\usepackage[a4paper]{geometry}
\usepackage{indentfirst}
\geometry{
    left=20mm,
    top=20mm,
}
\renewcommand{\thesection}{\Roman{section}}
\renewcommand{\thesubsection}{\Roman{subsection}}
\title{\LARGE \textbf {\huge{Penerapan Riset Operasional dalam Informatika \\}}
    Ditulis dalam \LaTeX
}

\author{
    M Rizqi R\\
    20051204034\\
    Teknik Informatika 2020 B\\
    Universitas Negeri Surabaya\\
}
\date{\today}

\begin{document}
    \maketitle
    \section{Gambaran Umum}
    Penelitian Operasional adalah penerapan metode, teknik ilmiah
    dan peralatan dalam masalah yang terjadi dalam operasi bisnis
    untuk menemukan solusi optimal. Dalam kehidupan sehari-hari,
    sering kali bertemu masalah dengan tingkat kesulitan 
    yang berbeda.

    Dalam peraturannya, hasil yang didapat haruslah optimal.
    Namun dalam prosesnya, dihadapkan beberapa hambatan,
    salah satunya adalah sumber daya yang terbatas. Sehingga
    dapat mencapai hasil yang optimal. Diperlukan satu atau lebih
    metode. Kumpulan metode atau sarana untuk memecahkan masalah
    disebut \textit{pencarian operasional.} Dengan penelitian, 
    tugas dapat dibuat untuk tujuan memperoleh hasil yang optimal
    dengan pertimbangan keterbatasan sumberdaya yang ada.

    \section{Studi Kasus}
    \subsection{Bahasa Pemrogramman Assembly}
    Pada awal perkembangan komputer, ukuran kapasitas penyimpanan pada 
    suatu sistem sangatlah terbatas. Program kebanyakan masih Ditulis
    dalam bahasa biner dengan basis (0 dan 1). Kemudian Kathleen Booth 
    bersama suaminya, Andrew Booth. Juga ditemani dengan seorang ahli 
    matematika, John von Neumann, ahli fisika Herman Goldstine melakukan
    penelitian untuk membuat sebuah bahasa pemrograman yang memudahkan 
    pengembang perangkat lunak.

    Akhirnya pada tahun 1948, bahasa Assembly diperkenalkan. Dengan fitur 
    \textit{memory management} yang bisa dilakukan secara manual, membuat 
    program yang ditulis dalam bahasa Assembly dapat dijalankan pada perangkat
    komputer yang pada saat itu semuanya memiliki kapasitas penyimpanan yang
    kecil. Ini memudahkan para programmer menulis program dan melakukan optimasi
    tanpa harus memindahkan setiap bit data secara manual. Syntax yang dimiliki
    oleh bahasa Assembly lebih \textit{human readable} daripada bahasa biner
    yang selama ini digunakan.

    Dengan adanya bahasa Assembly, proses pengembangan menjadi lebih cepat
    sehingga banyak perangkat lunak seperti game (\textit{Super Mario Bross}),
    \textit{Linux Kernel, MS-DOS, BSD} ditulis dalam bahasa Assembly.
    Bahasa ini juga menjadi cikal bakal bahasa C yang dikembangkan oleh
    Dennis Ritchie yang hingga saat ini masih digunakan untuk \textit{low-level
    programming.} 

    Hingga saat ini, bahasa Assembly masih digunakan untuk pengembangan perangkat
    lunak yang langsung bersentuhan dengan perangkat keras seperti \textit{kernel development,
    micro-controller, system-calls, dll.}

    
\end{document}