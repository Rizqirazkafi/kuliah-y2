\documentclass{article}
% Package Used
\usepackage[a4paper,top=30mm,left=40mm,bottom=30mm,right=30mm]{geometry}
\usepackage[immediate]{silence}
\usepackage{microtype}
% Title
\title{\textbf{Smoke Detector Pada Lab Kimia}}
\author{M. Rizqi R -  
        20051204034 -
        TI 2020 B \\
        \textbf{Universitas Negeri Surabaya}
}
\date{\today}

\begin{document}
    \maketitle

    \section{Pendahuluan}
    \subsection{Latar Belakang}
    Terjadinya kebocoran gas pada lab kimia sangat memungkinkan terjadi karena
    kecelakaan yang tak terduga. Hal ini dapat berakibat fatal apabila kecelakaan
    terjadi ketika lab tidak dalam status pengawasan atau kosong. Dibutuhkannya 
    sistem monitor yang mampu mendeteksi adanya kebocoran gas berbahaya atau asap 
    api untuk memberitahu pihak yang berwenang ketika terjadi kecelakaan tersebut.
    Dengan sistem monitor menggunakan Arduino Uno akan mempermudah proses tersebut
    dengan memberi pesan notifikasi kepada pihak yang berwenang terhadap lab
    maupun pengawas yang dekat dengan ruang lab.

    Metode ini dapat dicapai berkat adanya IoT (\textit{Internet of Things}) 
    yang membuat kita mampu mengendalikan perangkat keras pada jarak jauh
    menggunakan koneksi internet.
    \subsection{Rumusan Masalah}
    Dari latar belakang dapat dirumuskan masalah diantaranya:
    \begin{enumerate}
        \item Penjelasan tentang \textit{smoke detector}.
        \item Penjelasan tentang \textit{Internet of Things}.
        \item Mengapa \textit{smoke detector} dibutuhkan dalam lab?
        \item Mengetahui penggunaan \textit{smoke detector} pada lab.
    \end{enumerate} 
    \subsection{Batasan Masalah}
    Berdasarkan latar belakang penelitian dan rumusan masalah, maka dalam hal ini
    permasalahan yang dikaji perlu dibatasi. Pembatasan masalah ini bertujuan untuk
    memfokuskan perhatian pada penelitian dengan memperoleh kesimpulan yang benar
    dan mendalam pada aspek yang diteliti. Adapun pembatasan masalah yang dikaji
    dalam penelitian ini adalah: Masalah yang diteliti terbatas pada bagaimana 
    teknologi \textit{smoke detector} pada lab.

    \subsection{Tujuan Penelitian}
    \begin{enumerate}
        \item Memahami cara kerja \textit{smoke detector.}
        \item Mengetahui penggunaan \textit{smoke detector} pada lab.
        \item Mengetahui manfaat \textit{smoke detector} pada lab.
    \end{enumerate}
    
    \pagebreak

    \section{Pembahasan}
    \subsection{Penjelasan \textit{Internet of Things}}
    Internet of Things (IoT) adalah istilah yang digunakan
    untuk menggambarkan koneksi perangkat, mesin, dan
    instrumen di internet. Teknologi IoT memungkinkan
    pengguna berinteraksi dengan perangkat mereka dari
    jarak jauh. Istilah "Internet of Things" pertama kali
    diciptakan oleh Kevin Ashton pada tahun 1999.

    Ada tiga jenis utama implementasi IoT:
    \begin{enumerate}
        \item Perangkat Tertanam: Ini adalah sensor-sensor
            kecil yang dapat ditanamkan ke dalam objek atau
            perangkat. Sensor-sensor ini memonitor hal-hal
            seperti suhu, gerakan, tekanan dan getaran.
            Misalnya, sensor dapat ditempatkan pada bagian
            mesin untuk memantau seberapa besar tekanan
            yang telah diterapkan padanya dari waktu ke
            waktu.
        \item Sensor Dalam/Luar Ruangan: Sensor ini
            memantau hal-hal seperti suhu, tekanan dan
            kelembaban di dalam dan di luar ruangan. Sensor
            ini biasanya ditempatkan di dekat pintu atau
            jendela sehingga dapat melacak perubahan pola
            cuaca serta apakah jendela terbuka atau
            tertutup.
        \item Layanan Terkelola: Layanan terkelola adalah
            sistem berbasis cloud yang memungkinkan
            pengguna memantau perangkat mereka yang
            terhubung dari jarak jauh dan menerima
            peringatan ketika terjadi kesalahan pada
            perangkat tersebut.
    \end{enumerate} 
    \subsection{Penjelasan NodeMCU ESP-8266}
    ESP8266 adalah mikrokontroler yang sangat fungsional yang
    dapat berjalan pada berbagai sumber input yang berbeda.
    Mikrokontroler ini memiliki kemampuan Wi-Fi built-in dan
    mampu berjalan sebagai server web, yang berarti dapat
    mengontrol perangkat dari jarak jauh. ESP8266 juga mampu
    mengendalikan perangkat lain, seperti LED, motor, dan
    sensor.

    \subsection{Sensor \textit{Smoke Detector}}
 

    Detektor asap dan gas adalah alat yang mendeteksi adanya
    asap dan/atau gas di lingkungan. Hal ini dapat dilakukan
    dengan menggunakan beberapa metode yang berbeda, termasuk:
    \begin{enumerate}
        \item Ionisasi - Deteksi partikel terionisasi dalam medan
             listrik. Metode ini memerlukan detektor dengan bahan
             radioaktif (yaitu Americium 241) di dalamnya, yang
             menghasilkan ion ketika terkena asap atau gas.

        \item Photoelectric - Menggunakan cahaya untuk mendeteksi
                partikel asap, yang menyerap cahaya pada panjang gelombang
                yang berbeda tergantung pada ukuran partikel. Detektor
                menyinari cahaya melalui filter yang hanya memungkinkan
                panjang gelombang tertentu untuk melewatinya (mis: biru).
                Jika ada partikel di udara yang menyerap panjang gelombang
                tersebut, mereka akan tampak gelap terhadap latar belakang
                terang cahaya yang disaring (misal: hitam).
    \end{enumerate}
    \section{Penutup}
    \subsection{Kesimpulan}
    Sebagai kesimpulan, penting untuk dicatat bahwa fungsi
    deteksi asap IoT di laboratorium kimia tidak terbatas
    hanya untuk memperingatkan orang-orang ketika kebakaran
    telah terjadi. Teknologi ini juga dapat digunakan untuk
    memonitor suhu dan tingkat kelembapan di laboratorium.
    Dengan data ini, para peneliti dapat lebih memahami
    bagaimana keadaan yang berbeda memengaruhi reaksi yang
    terjadi di dalam laboratorium mereka. Informasi ini
    dapat membantu para ahli kimia membuat keputusan yang
    lebih baik tentang bahan apa yang harus mereka gunakan,
    berapa banyak dari setiap bahan yang perlu mereka
    gunakan, dan bagaimana mereka harus mengatur peralatan
    mereka.

    \begin{thebibliography}{3}
        \bibitem[1]A. Abdullah, “DETECTION AND MONITORING SYSTEM OF SMOKE CONCENTRATION WITH SMOKE DETECTOR AND CAMERA TRACKER,” FISITEK, vol. 2, no. 1, p. 1, Feb. 2018, doi: 10.30821/fisitek.v2i1.1433.
        \bibitem[2]M. R. Hidayat, C. Christiono, and B. S. Sapudin, “PERANCANGAN SISTEM KEAMANAN RUMAH BERBASIS IoT DENGAN NodeMCU ESP8266 MENGGUNAKAN SENSOR PIR HC-SR501 DAN SENSOR SMOKE DETECTOR,” kilat, vol. 7, no. 2, pp. 139–148, Nov. 2018, doi: 10.33322/kilat.v7i2.357.
        \bibitem[3]M. S. Mohamad Adenan, A. N. Mas Erwan, and M. N. H. Muzaffar Alfian, “Smart Smoke Detector,” ijortas, vol. 3, no. 1, pp. 16–31, Mar. 2021, doi: 10.36079/lamintang.ijortas-0301.198.
        
    \end{thebibliography}

\end{document}
