\documentclass{article}

\usepackage{indentfirst}
\usepackage{amsmath}
\usepackage{graphicx}
\usepackage{hyperref}
\usepackage[utf8]{inputenc}
\usepackage{lipsum}
\usepackage[a4paper]{geometry}
\geometry{
    left=20mm,
    top=20mm,
}
\renewcommand{\thesection}{\Roman{section}}
\renewcommand{\thesubsection}{\Roman{subsection}}
\title{\LARGE \textbf {\huge{Mendalami Onion Network dengan TOR}}}
\author{
    M Rizqi R\\
    20051204034\\
    Teknik Informatika 2020 B\\
    Universitas Negeri Surabaya\\
    \and
        Alfito Mulyono\\
        20051204038\\
        Teknik Informatika 2020 B\\
        Universitas Negeri Surabaya\\
}
\date{\today}

\begin{document}
\pagenumbering{gobble}
    \maketitle
        \section{Pendahuluan}
        \subsection*{1.1 Latar Belakang}
        Bocornya informasi pengguna internet memiliki dampak yang merugikan.
        Banyak sekali pihak yang ingin mencuri data kita untuk 
        dimanfaatkan dalam kepentingan mereka pribadi. Mereka akan menjual 
        data kita kepada pihak-pihak yang menginginkan data kita.

        Kita dapat mengambil contoh dari sebuah perusahaan media sosial ternama
        seperti Facebook. Pada 19 Desember 2018, Facebook diketahui telah
        membocorkan 1,5 miliar data pengguna Facebook dan menjualnya kepada pihak ketiga.
        Mark Zuckerberd, pendiri Facebook dinyatakan bersalah dalam persidangan 
        dan dijatuhi denda sesuai keputusan hakim.

        Namun kebocoran data bisa juga terjadi karena kesalahan pengguna.
        Pengguna kerap mengunjungi situs-situs acak kemudian tanpa
        sengaja menekan iklan yang muncul dalam situs tersebut. 
        Kemudian pengguna tanpa sadar telah mengunduh perangkat
        lunak yang berisi virus yang kemudian berjalan dibelakang layar.
        Virus ini dapat berupa berbagai jenis perangkat lunak yang dapat
        menginfeksi perangkat pengguna.

        Ada pula virus yang bahkan bisa digunakan untuk melacak setiap jengkal
        aktivitas pengguna. Baik itu dari ketikan keyboard, lokasi, bahkan mampu memanipulasi
        berkas digital yang dimiliki oleh pengguna hingga bisa mencuri berkas tersebut.

        Maka dari itu diperlukan cara agar kita terhindar dari kejahatan internet tersebut.
        Salah satunya ialah memilih alat penjelajah web (\textit{web browser}) yang tepat.
        Web browser yang tidak melacak kita saat menjelajah internet dan mencegah
        situs-situs berbahaya melacak dan mencuri data kita.
        \subsection*{1.2 Tujuan}
        Berdasarkan latar belakang yang telah dipaparkan. Tujuan makalah ini adalah
        untuk memberikan wawasan tentang \textit{web browser} yang fokus pada 
        privasi dan keamanan pengguna.

        \section{Rumusan Masalah}
        \section{Penutup}
\pagenumbering{arabic}
\end{document}