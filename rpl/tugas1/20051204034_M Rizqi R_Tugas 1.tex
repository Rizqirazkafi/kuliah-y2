\documentclass{article}
\usepackage{indentfirst}
\usepackage{amsmath}
\usepackage{graphicx}
\usepackage{hyperref}
\usepackage[utf8]{inputenc}
\usepackage{lipsum}
\usepackage[a4paper]{geometry}
\usepackage{xcolor}
\usepackage{multicol}
\geometry{
    left=20mm,
    top=20mm,
}

\title{\Large \textbf{\huge{Resume Pengembangan Perangkat Lunak}}}
\author{
    M. Rizqi R\\
    20051204034\\
    Teknik Infromatika 2020 B\\
}
\date{\today}
\begin{document}
    \maketitle
    \begin{multicols*}{2}
    \section*{A. Pengertian}
    Dalam setiap pengembangan perangkat lunak, pengembang perlu memastikan
    setiap aspek dari sistem dirancang dengan baik untuk menjamin kualitas
    perangkat lunak yang dibuat.Untuk membantu pemngembang merancang aplikasi dengan kualitas
    yang baik, tersedia proses SDLC.

    SDLC (Software Development Life Cycle) adalah proses yang digunakan
    pengembang untuk merancang, menguji, dan mengembangkan software dengan
    kualitas tinggi. SDLC menyediakan alur terstruktur yang membantu pengembang
    enghasilkan software berkualiras tinggi dengan waktu lebih singkat dan biaya yang
    lebih murah, namun dapat memenuhi atau melampaui ekspektasi pelanggan.
    Tahapan pengembangan SDLC :
    
    \begin{enumerate}
        \item Planning
        \item Define requirements
        \item Design and prototyping
        \item Software development
        \item Testing
        \item Deployment
        \item Operations and maintenance
    \end{enumerate}
    \section*{B. Model Pengembangan SDLC}
    Ada beberapa metode yang dapat digunakan dalam pengembangan SDLC. 
    Model-model tersebut adalah sebagai berikut:
    \subsection*{1. Metode Waterfall}
    Metode waterfall merupakan salah satu metode SDLC paling tua. Juga dikenal
    sebagai metode paling mudah dengan menyelesaikan satu fase secara total
    kemudian dilanjutkan ke fase berikutnya tanpa perlu mengulang.

    Setiap fase dari model ini bergantung pada informasi dari fase sebelumnya
    juga rencana proyek yang telah kita rancang. Apabila satu fase belum 
    diselesaikan secara maksimal, maka kita akan sulit untuk melanjutkan
    ke fase berikutnya. Model ini mudah dipahami, namun datang dengan 
    satu kekurangan yaitu waktu.
    Apabila dalam satu fase memakan waktu lama, maka akan menghambat
    proses pengembangan di fase berikutnya. Ini dapat berakibat
    perubahan garis waktu pada keseluruhan proyek.

    \subsection*{2. Metode Spiral}
    Metode spiral merupakan metode SDLC yang paling 
    fleksibel. Metode ini dilakukan dengan pengulangan 
    dan pengulangannya melewati empat fase yang diulang 
    mengikuti pola spiral pada fase-fasenya hingga 
    selesai. Metode ini memungkinkan adanya penyempurnaan 
    pada setiap fase.

    Dengan metode ini, pembuatan perangkat lunak dapat 
    disesuaikan dengan feedback client, baik di awal 
    maupun pertengahan proyek.
    Kekurangan dari metode ini adalah resiko bahwa
    spiral dalam proses pengembangan tidak pernah 
    berakhir karena proyek selalu diperbarui.

    \subsection*{3. Prototype Model}
    Metode pengembangan yang memungkinkan pengguna
    membuat gambaran awal tentang program yang akan dibuat.
    Dimulai dari menetapkan kebutuhan perangkat lunak yang 
    dilakukan client dan developer bersama-sama 
    kemudian membuat rancangan sesuai kebutuhan yang telah
    ditetapkan (prototype) dan mengevaluasi prototype tersebut.
    Selanjutnya prototype diterjemahkan kedalam bahasa
    pemrograman yang sesuai. Setelah proses pembuatan selesai
    dan sistem sudah menjadi perangkat lunak, maka dilakukan
    testing. Sisanya adalah evaluasi dari client 
    dan program siap digunakan.
        
    \end{multicols*}
\end{document}